\chapter{\ifcpe ทฤษฎีที่เกี่ยวข้อง\else Background Knowledge and Theory\fi}
\label{ch:background}

เนื้อหาในบทนี้จะอธิบายเกี่ยวกับเทคโนโลยีต่างๆ ที่นำมาประยุกต์ใช้ในโครงงาน Correg

\section{Terminology}

\subsection{VM instances}

Virtual machines (VM) \cite{vm}\CIreply{ควรมีหรือไม่ควรมีคำว่า instances?} คือ เทคโนโลยีคอมพิวเตอร์เสมือน ที่จำลองคอมพิวเตอร์ขึ้นมา โดยใช้ซอฟต์แวร์เพื่อจำลองการทำงานของคอมพิวเตอร์เครื่องเดียวหรือหลายเครื่อง ไว้บนคอมพิวเตอร์จริง (physical hardware) เพียงเครื่องเดียว

\subsection{Kubernetes cluster}

Kubernetes cluster คือ กลุ่มของเครื่องเซิฟเวอร์ รวมถึง VM ที่รัน Docker และนำมาทำงานร่วมกันผ่าน Kubernetes control plane \cite{kubecomp} ทำให้เครื่องเซิฟเวอร์ทำงานร่วมกันได้

\subsection{Framework}

Framework \cite{framework} คือ library สำหรับกำหนดโครงสร้างของโปรแกรม เพื่อให้ง่ายต่อการพัฒนาต่อ และสามารถนำโครงสร้างของโปรแกรมไปใช้ต่อได้ทันที

\section{Infrastruture}

\subsection{Docker}

Docker \cite{docker} คือ engine ตัวหนึ่งที่มีการทำงานเป็น OS virtualization \cite {osvirtual} เพื่อใช้สำหรับการการรัน server ต่างๆ โดยจะต้องทำให้โปรแกรมนั้นอยู่ในรูปแบบของ image และเมื่อนำไปทำงานจะกลายเป็น container ซึ่งจะทำให้ง่ายต่อการจัดเก็บและนำไปใช้ในระดับ production

\subsection{Kubernetes}

Kubernetes \cite{kubernetes} คือ โปรแกรมที่ทำหน้าที่เป็น orchestration \cite  {orchestration} ของ Docker ช่วยให้จัดการระบบเครือข่ายระหว่าง containers ได้ง่ายขึ้น
\CIreply{บอกด้วยว่าเทคโนโลยีที่นำมาใช้แต่ละอย่าง จะช่วยแก้ปัญหาตรงจุดไหน}

\section{Tools}

\subsection{Nodejs}

Nodejs \cite{nodejs} คือ tools สำหรับ compile JavaScript \cite{javascript} เพื่อให้โปรแกรมนั้นสามารถทำงานได้โดยไม่ต้องพึ่ง browser engine ในการทำงาน

\section{Database}
\subsection{MySQL}

MySQL เป็นระบบฐานข้อมูลในรูปแบบ SQL และฐานข้อมูลนี้มีการเก็บรักษาข้อมูลโดยใช้หลักการของ ACID model \cite{acid}\CIreply{คืออะไร อธิบาย}

\section{Backend}

\subsection{Nestjs}

Nestjs \cite{nestjs} คือ \CI{nodejs}{check capitalization} framework ไว้สำหรับพัฒนา server application \cite {serverapplication} พื้นฐานของ framework ถูกออกแบบมาให้มีหลักการทำงานคล้ายๆ กับ Angular framework \cite{angular} ที่มีความนิยมในระดับหนึ่ง
\CIreply{ข้อดีของ Nestjs คืออะไร ทำไมถึงเลือกใช้แทนที่จะเลือกอย่างอื่น  ตัวเลือกอื่นมีอะไรบ้าง}

\section{\ifcpe%
ความรู้ตามหลักสูตรซึ่งถูกนำมาใช้หรือบูรณาการในโครงงาน
\else%
ISNE knowledge used, applied, or integrated in this project
\fi
}

\begin{enumerate}
    \item Parallel computing
    \item Database design
    \item Object-oriented programming
\end{enumerate}

\section{\ifcpe%
ความรู้นอกหลักสูตรซึ่งถูกนำมาใช้หรือบูรณาการในโครงงาน
\else%
Extracurricular knowledge used, applied, or integrated in this project
\fi
}

\begin{enumerate}
    \item Docker concepts
    \item Kubernetes concepts
    \item Caching concepts\CIreply{ไม่ได้อยู่ใน OS?}
\end{enumerate}
