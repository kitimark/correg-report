\chapter{\ifcpe ทฤษฎีที่เกี่ยวข้อง\else Background Knowledge and Theory\fi}
\label{ch:background}

เนื้อหาในบทนี้จะอธิบายเกี่ยวกับเทคโนโลยีต่างๆ ที่นำมาประยุกต์ใช้ในโครงงาน Correg

\section{Terminolgy}

\subsection{VM Instances}

VM Instances \cite{vm} คือ เทคโนโลยีคอมพิวเตอร์เสมือนที่จำลองคอมพิวเตอร์ขึ้นมาให้สามารถใช้ซอฟต์แวร์เพื่อจำลองการทำงานของคอมพิวเตอร์เครื่องอื่นไว้ในเครื่องเดียว หรือหลายเครื่องบนคอมพิวเตอร์เครื่องจริง (Hardware) เพียงเครื่องเดียว

\subsection{Kubernetes Cluster}

Kubernetes Cluster คือ กลุ่มของเครื่องเซิฟเวอร์รวมถึง VM ที่รัน Docker และนำมาทำงานร่วมกันผ่าน Kubernetes control plane \cite{kubecomp} ทำให้เครื่องเซิฟเวอร์ทำงานร่วมกันได้

\subsection{Framework}

Framework \cite{framework} คือ library สำหรับกำหนดโครงสร้างขอโปรแกรมที่เพื่อให้ง่ายต่อการพัฒนาต่อ และสามารถนำโครงสร้างขอโปรแกรมไปใช้ต่อได้ทันที

\section{Infrastruture}

\subsection{Docker}

Docker \cite{docker} คือ engine ตัวหนึ่งที่มีการทำงานเป็น OS Virtualization \cite {osvirtual} เพื่อใช้สำหรับการการรัน server ต่างๆ โดยจะต้องทำให้โปรแกรมนั้นอยู่ในรูปแบบของ image และเมื่อนำไปทำงานจะกลายเป็น container ซึ่งจะทำให้ง่ายต่อการจัดเก็บและนำไปใช้ในระดับ production

\subsection{Kubernetes}

Kubernetes \cite{kubernetes} คือ โปรแกรมที่ทำหน้าที่เป็น orchestration \cite  {orchestration} ของ docker ช่วยให้จัดการ ระบบเครือข่ายระหว่าง container ได้ง่ายขึ้น

\section{Tools}

\subsection{Nodejs}

Nodejs \cite{nodejs} คือ tools สำหรับ compile JavaScript \cite{javascript} เพื่อให้โปรแกรมนั้นสามารถทำงานได้โดยไม่่ต้องพึ่ง browser engine ในการทำงาน

\subsection{DataLoader}

DataLoader \cite{dataloader} คือ utility ที่เป็นส่วนหนึ่งของ layer ของการดึงข้อมูลที่ช่วยให้จัดการดึงข้อมูลต่างๆ ได้อย่างมีประสิทธิภาพ


\section{Database}
\subsection{MySQL}

MySQL เป็นระบบฐานข้อมูลในรูปแบบ SQL และฐานข้อมูลนี้มีการเก็บรักษาข้อมูลโดยใช้หลักการของ ACID Model \cite{acid}

\section{Frontend}

\subsection{Reactjs}

Reactjs \cite{reactjs} คือ tools ที่ไว้ใช้สำหรับพัฒนา application ฝั่ง frontend ซึ่งสามารถช่วยทำให้ พัฒนา application ได้ง่ายขึ้น ซึ่งจะ compile ให้กลายเป็น JavaScript ก่อนที่จะนำไปใช้ในระดับ production

\subsection{Nextjs}

Nextjs \cite{nextjs} คือ framework ที่พัฒนาเพี่อรองรับ Reactjs ซึ่ง framework นี้สามารถเพิ่มประสิทธิภาพของ Reactjs เหมาะสำหรับ web application ขนาดใหญ่สามารถทำ CDN \cite{cdn} ได้ง่าย

\section{Backend}

\subsection{Nestjs}

Nestjs \cite{nestjs} คือ nodejs framework ไว้สำหรับพัฒนา server application \cite {serverapplication} พื้นฐานของ framework ถูกออกแบบมาให้มีหลักการทำงานคล้ายๆ กับ Angular framework \cite{angular} ที่มีความนิยมในระดับหนึ่ง

\subsection{GraphQL}

GraphQL \cite{graphql} คือ web-based API \cite{webapi} ที่มีความสามารถลดจำนวน request ที่เกิดขึ้นในระหว่าง server กับ client และสามารถ เพิ่มประสิทธิการเข้าถึงข้อมูล และการทำงานในระดับ server ได้ โดยที่ GraphQL จะสามารถบอกได้ว่า client ต้องการอะไรบ้าง และฝั่ง server ก็จะสามารถทำงานตามเฉพาะที่ฝั่ง client ต้องการเท่านั้น

\section{\ifcpe%
ความรู้ตามหลักสูตรซึ่งถูกนำมาใช้หรือบูรณาการในโครงงาน
\else%
ISNE knowledge used, applied, or integrated in this project
\fi
}

\begin{enumerate}
    \item Parallel computing
    \item Database design
    \item Object-oriented programming
\end{enumerate}

\section{\ifcpe%
ความรู้นอกหลักสูตรซึ่งถูกนำมาใช้หรือบูรณาการในโครงงาน
\else%
Extracurricular knowledge used, applied, or integrated in this project
\fi
}

\begin{enumerate}
    \item Docker concept
    \item Kubernetes concept
    \item Caching concept
\end{enumerate}
