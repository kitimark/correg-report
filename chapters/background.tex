\chapter{\ifcpe ทฤษฎีที่เกี่ยวข้อง\else Background Knowledge and Theory\fi}

การทำโครงงาน เริ่มต้นด้วยการศึกษาค้นคว้า ทฤษฎีที่เกี่ยวข้อง หรือ งานวิจัย/โครงงาน ที่เคยมีผู้นำเสนอไว้แล้ว ซึ่งเนื้อหาในบทนี้ก็จะเกี่ยวกับการอธิบายถึงสิ่งที่เกี่ยวข้องกับโครงงาน เพื่อให้ผู้อ่านเข้าใจเนื้อหาในบทถัดๆ ไปได้ง่ายขึ้น

\section{Terminolgy}

\subsection{Framework}

Framework \cite{framework} คือ library สำหรับกำหนดโครงสร้างขอโปรแกรมที่เพื่อให้ง่ายต่อการพัฒนาต่อ และสามารถนำโครงสร้างขอโปรแกรมไปใช้ต่อได้ทันที

\section{Operating System}

\subsection{Docker}

\enskip Docker \cite{docker} คือ engine ตัวหนึ่งที่มีการทำงานเป็น OS Virtualization \cite {osvirtual} เพื่อใช้สำหรับการการรัน server ต่างๆ โดยจะต้องทำให้โปรแกรมนั้นอยู่ในรูปแบบของ image และเมื่อนำไปทำงานจะกลายเป็น container ซึ่งจะทำให้ง่ายต่อการจัดเก็บและนำไปใช้ในระดับ production

\section{Tools}

\subsection{Nodejs}

\enskip Nodejs \cite{nodejs} คือ tools สำหรับ compile javascript \cite{javascript} เพื่อให้โปรแกรมนั้นสามารถทำงานได้โดยไม่่ต้องพึ่ง browser engine ในการทำงาน

\subsection{Kubernetes}

\enskip Kubernetes \cite{kubernetes} คือ โปรแกรมที่ทำหน้าที่เป็น orchestration \cite  {orchestration} ของ docker ช่วยให้จัดการ ระบบเครือข่ายระหว่าง container ได้ง่ายขึ้น

\subsection{DataLoader}

\enskip DataLoader \cite{dataloader} คือ ulitity ที่เป็นส่วนหนึ่งของ layer ของการดึงข้อมูลที่ช่วยให้จัดการดึงข้อมูลต่างๆ ได้อย่างมีประสิทธิภาพ


\section{ระบบฐานข้อมูล}
\subsection{MySQL}

\enskip MySQL เป็นระบบฐานข้อมูลในรูปแบบ SQL และฐานข้อมูลนี้มีการเก็บรักษาข้อมูลโดยใช้หลักการของ ACID Model \cite{acid}

\section{Frontend}

\subsection{Reactjs}

\enskip Reactjs \cite{reactjs} คือ tools ที่ไว้ใช้สำหรับพัฒนา application ฝั่ง frontend ซึ่งสามารถช่วยทำให้ พัฒนา application ได้ง่ายขึ้น ซึ่งจะ compile ให้กลายเป็น javascript ก่อนที่จะนำไปใช้ในระดับ production

\subsection{Nextjs}

\enskip Nextjs \cite{nextjs} คือ framework \cite{framework} ที่พัฒนาเพี่อรองรับ Reactjs \cite{reactjs} ซึ่ง framework นี้สามารถเพิ่มประสิทธิภาพของ React \cite{reactjs} เหมาะสำหรับ Web application ขนาดใหญ่สามารถทำ cdn \cite{cdn} ได้ง่าย

\section{Backend}

\subsection{Nestjs}

\enskip Nestjs \cite{nestjs} คือ nodejs \cite{nodejs} framework \cite{framework} ไว้สำหรับพัฒนา server application \cite {serverapplication} พื้นฐานของ framework ถูกออกแบบมาให้มีหลักการทำงานคล้ายๆ กับ Angular framework \cite{angular} ที่มีความนิยมในระดับหนึ่ง

\subsection{GraphQL}

\enskip GraphQL \cite{graphql} คือ web-base api \cite{webapi} ที่มีความสามารถลดจำนวน request ที่เกิดขึ้นในระหว่าง server กับ client และสามารถ เพิ่มประสิทธิการเข้าถึงข้อมูล และการทำงานในระดับ server ได้ โดยที่ GraphQL จะสามารถบอกได้ว่า client ต้องการอะไรบ้าง และฝั่ง server ก็จะสามารถทำงานตามเฉพาะที่ฝั่ง client ต้องการเท่านั้น

\section{\ifcpe%
ความรู้ตามหลักสูตรซึ่งถูกนำมาใช้หรือบูรณาการในโครงงาน
\else%
ISNE knowledge used, applied, or integrated in this project
\fi
}

อธิบายถึงความรู้ และแนวทางการนำความรู้ต่างๆ ที่ได้เรียนตามหลักสูตร ซึ่งถูกนำมาใช้ในโครงงาน

\section{\ifcpe%
ความรู้นอกหลักสูตรซึ่งถูกนำมาใช้หรือบูรณาการในโครงงาน
\else%
Extracurricular knowledge used, applied, or integrated in this project
\fi
}

อธิบายถึงความรู้ต่างๆ ที่เรียนรู้ด้วยตนเอง และแนวทางการนำความรู้เหล่านั้นมาใช้ในโครงงาน
