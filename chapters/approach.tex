\chapter{\ifproject%
\ifcpe โครงสร้างและขั้นตอนการทำงาน\else Project Structure and Methodology\fi
\else%
\ifcpe โครงสร้างของโครงงาน\else Project Structure\fi
\fi
}

ในบทนี้จะกล่าวถึงหลักการ และการออกแบบระบบ ซึ่งเราจะออกแบบระบบลงทะเบียนที่มีอยู่แล้วให้มีประสิทธิภาพมากขึ้น

\makeatletter

\section {Infrastructure}

จากที่มาของโครงงานในบทที่ 1 เราได้กล่าวถึงโครงสร้างพื้นฐานของเซิฟเวอร์ที่สำนักทะเบียนในปัจจุบันมีการแบ่ง VM \cite{vm} ได้ไว้อย่างไม่มีประสิทธิภาพ เนื่องจาก VM ชุดนี้ไม่ได้มีการใช้ System center orchestrator \cite{sco} ดังนั้น เราไม่สามารถทำการ deploy โปรแกรมให้เหมาะกับจำนวนผู้ใช้งานในระยะเวลาหนึ่ง และหรือถ้าโปรแกรมนั้นมีจำนวนผู้ใช้มากในระดับที่ VM ที่อยู่ไม่มีทรัพยากรเพืยงพอที่จะรองรับ ตัวของโครงสร้างพื้นฐานนี้เองก็ไม่สามารถปรับการ deploy โปรแกรมนั้นๆ ไปที่ VM อื่นๆ ที่มีทรัพยากรเหลือที่ให้ใช้งานซึ่งเป็นข้อเสียหลักๆ ที่ทำให้ระบบล่มในเวลาต่อมา

จากรูป ??? เราได้วางแผนโครงสร้างพื้นฐานใหม่ โดยการนำ VM ทั้งหมดมารวมกันโดยทำให้อยู่ในรูปแบบ Kubernetes cluster ซึ่งสามารถทำให้ VM ต่างๆ สามารถทำงานร่วมกันได้ผ่าน Kubernetes control plane \cite{kubecomp} ดังนั้นเราสามารถ deploy ตัวของโปรแกรมต่างๆ ที่ต้องการให้ทำงานพร้อมทั้งกำหนดการขยายตัวของจำนวน containers ได้จากการวัดการใช้่งานโปรแกรมใน container นั้นๆ เช่น การใช้งานของตัวประมวลผล, การใช้งานของหน่วยความจำชั่วคราว หรือจำนวนการร้องขอ การใช้งานของตัวโปรแกรมนั้น หรืออื่นๆ ที่เราสามารถกำหนดได้เองผ่าน Horizontal Pod Autoscaler \cite{kubehpa} ของ Kubernetes ได้ แล้วทางตัวของ Kubernetes จะทำการคำนวณและทำการ deploy ตัวของ container ให้อัตโนมัติลงไปใน VM ที่ยังคงมีทรัพยากรเหลือใช้อยู่