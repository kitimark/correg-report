\chapter{\ifproject%
\ifcpe การทดลองและผลลัพธ์\else Experimentation and Results\fi
\else%
\ifcpe การประเมินระบบ\else System Evaluation\fi
\fi}

\section{ด้านความถูกต้อง}

\subsection{Infrastructure}

\subsubsection{การทำงานร่วมกันระหว่างโปรแกรม}

โปรแกรมที่กล่าวถึงก่อนหน้านี้ ที่จำเป็นทีจะต้องย้ายเข้าระบบใหม่ จะต้องสามารถทำงานได้อย่างปกติ และยังสามารถติดต่อเพื่อเรียกใช้งานได้ เช่น เข้าถึงเว็บไซต์ ค้นหาวิชาเรียน ดูข้อมูลของสำนักทะเบียนได้เหมือนเดิม เป็นต้น

\subsection{Enrollment system}

\subsubsection{การลงทะเบียน}

ระบบจะต้องสามารถลงทะเบียนได้คล้ายเดิม หลังการปรับแต่งในส่วนของการติดต่อข้อมูลระหว่าง frontend และ backend ได้

\subsubsection{การจัดเก็บข้อมูลในระบบลงทะเบียน}

ฐานข้อมูลที่จัดเก็บข้อมูลการลงทะเบียนจะต้องมีความซ้ำซ้อนน้อยลงกว่าระบบเดิมที่มีอยู่แล้ว โดยสามารถวัดผลโดยการ ดูจากโครงสร้างของการเก็บข้อมูลในฐานข้อมูลนั้นมีขนาดที่เล็กลงกว่า แบบการเก็บข้อมูลลงทะเบียนของสำนักทะเบียนที่มีใช้งานอยู่ในปัจจุบัน

\subsubsection{โปรแกรมฝั่งเซิฟเวอร์}

โปรแกรมจะต้องสามารถติดต่อฐานข้อมูลและสามารถรวบรวมข้อมูล (aggregation) หลายๆ ฐานข้อมูลและส่งไปแสดงผลให้กับฝั่งของ frontend ได้อย่างถูกต้อง

\subsubsection{การทำงานร่วมกันระหว่าง frontend และ backend}

โปรแกรมทั้งสองฝั่งต้องสามารถติดต่อและทำงานรวมกันได้อย่างถูกต้อง

\section{ด้านประสิทธิภาพ}

\subsection{Infrastructure}

\subsubsection{การรองรับผู้ใช้งาน}

โปรแกรมที่กล่าวถึงก่อนหน้านี้ ที่จำเป็นทีจะต้องย้ายเข้าระบบใหม่ จะต้องสามารถรับการใช้งานเพิ่มขึ้นในขณะหนึ่งๆ ด้วยการขยายตัวหรือหดตัวตามความเหมาะสมได้

\subsection{Enrollment system}

\subsubsection{SQL statements}

เราไม่สามารถวัดผลในระดับ SQL statements ของการเรียกขอการใช้งานระดับโปรแกรมฝั่งเซิฟเวอร์ได้ เนื่องจากในระบบที่เราจะทำการพัฒนานั้น layer ที่ทำการเรียกขอข้อมูลจะทำงานไม่เหมือนกับโปรแกรมฝั่งเซิฟเวอร์ของสำนักทะเบียนที่มีอยู่ในปัจจุบัน

\subsubsection{การรองรับผู้ใช้งาน}

โปรแกรมในฝั่งเซิฟเวอร์ที่เราพัฒนาจะต้องสามารถรองรับผู้ใช้งานได้เพิ่มขึ้นจากโปรแกรมในฝั่งเชิฟเวอร์ของสำนักทะเบียนที่มีอยู่แล้ว โดยการวัดผล จะวัดจากการร้องขอข้อมูลผ่านมาตรฐานการร้องขอข้อมูลนั้นๆ แต่จะต้องมีผลลัพธ์ที่เหมือนกัน
