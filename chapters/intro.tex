\chapter{\ifcpe บทนำ\else Introduction\fi}

\section{\ifcpe ที่มาของโครงงาน\else Project rationale\fi}

\enskip ทางผู้พัฒนาเคยใช้โปรแกรมลงทะเบียนวิชาเรียนมาเป็นเวลา 3 ปี ได้พบถึงปัญหาการใช้งานต่างๆ เช่น เว็บไซต์สำหรับการลงทะเบียนวิชาเรียนเกิดการล่ม รวมถึงเว็บไซต์หลักของสำนักทะเบียน ทำให้ไม่สามารถเข้าถึงเว็บไวต์ลงทะเบียนวิชาโดยง่าย หรือ ตัวของระบบลงทะเบียนเองที่มีช่องโหว่อยู่หลายแห่ง ซึ่งสามารถส่งผลกระทบต่อระบบโดยรวมได้

หลังจากที่เราได้สัมภาษณ์ ผู้พัฒนาโปรแกรมของ สำนักทะเบียนเอง ทำไห้เราได้พบ ปัญหาอีกหลายๆ อย่างที่เกิดขึ้น เช่น การแบ่งตัวของ VM \cite{vm} มาใช้อย่างไม่มีประสิทธิภาพที่ควรจะเป็น หรือมีข้อมูลในฐานข้อมูลที่ซ้ำซ้อน ในหลายๆ ฐานข้อมูล ซึ่งอาจจะส่งผลกระทบในเรื่องของ data inconsistency ได้ในภายหลัง

เราจึงจะพัฒนาระบบลงทะเบียนที่มีสามารถรองรับการใช้งานตามจำนวนของนักศึกษาที่ต้องการลงทะเบียน และ ระบบนี้สามารถรักษา data consistency โดยการทำให้ หลายๆ ฐานข้อมูลสามารถทำงานร่วมกันได้

\section{\ifcpe วัตถุประสงค์ของโครงงาน\else Objectives\fi}
\begin{enumerate}
    \item เพิ่มประสิทธิภาพของระบบลงทะเบียน
    \item ระบบสามารถรองรับการใช้งานของจำนวนนักศึกษาที่ต้องการเข้าใช้ในระบบ
    \item อุดช่องโหว่ของระบบที่เกิดขึ้น
    \item ลดความซ้ำซ้อนของข้อมูลในฐานข้อมูล
\end{enumerate}

\section{\ifcpe ขอบเขตของโครงงาน\else Project scope\fi}

\subsection{\ifcpe ขอบเขตด้านฮาร์ดแวร์\else Hardware scope\fi}
\begin{enumerate}
    \item PC ที่สามารถเข้าถึง browser เพื่อทดสอบ GraphQL API
\end{enumerate}
\subsection{\ifcpe ขอบเขตด้านซอฟต์แวร์\else Software scope\fi}
\begin{enumerate}
    \item ฐานข้อมูลต้องเพียงพอรองรับต่อการลงทะเบียน
    \item สามารถใช้งานร่วมกับ frontend ที่ทำหน้าที่แสดงผลการลงทะเบียน
\end{enumerate}
\section{\ifcpe ประโยชน์ที่ได้รับ\else Expected outcomes\fi}
\begin{enumerate}
    \item ระบบลงทะเบียนสามารถตอบรับตามจำนวนนักศึกษาที่ต้องการได้
    \item ระบบลงทะเบียนสามารถใช้ประสิทธิภาพของ server ได้สูงสุด
    \item ไม่มีข้อมูลที่ซ้ำซ้อน (ไม่ต้อง update ข้อมูลหลายๆ ที่)
\end{enumerate}

\section{\ifcpe เทคโนโลยีและเครื่องมือที่ใช้\else Technology and tools\fi}

เทคโนโลยีและเครื่องมือที่ใช้ที่กำลังจะกล่าวถึงด้านล่าง จะอธิบายเพิ่มเติมต่อในบทที่ 2

\subsection{\ifcpe เทคโนโลยีด้านฮาร์ดแวร์\else Hardware technology\fi}
\begin{enumerate}
    \item Kubernetes Cluster
    \item VM Instances
\end{enumerate}

\subsection{\ifcpe เทคโนโลยีด้านซอฟต์แวร์\else Software technology\fi}
\begin{enumerate}
    \item Nextjs \cite{nextjs}
    \item Reactjs \cite{reactjs}
    \item Nestjs \cite {nestjs}
    \item Nodejs \cite {nodejs}
    \item DataLoader \cite {dataloader}
    \item GraphQL \cite {graphql}
    \item MySQL \cite {mysql}
    \item Docker \cite {docker}
    \item Kubernetes \cite {kubernetes}
\end{enumerate}

\section{\ifcpe แผนการดำเนินงาน\else Project plan\fi}

\begin{plan}{6}{2020}{2}{2021}
    \planitem{6}{2020}{8}{2020}{ศึกษาการเปลี่ยนถ่ายระบบ (software transition) และกรณีศึกษา}
    \planitem{7}{2020}{7}{2020}{ศึกษาการวางแผนการขยายตัวของ pod ใน kubernetes เพื่อรองรับการขยายตัวของผู้ใช้งานระยะสั้น}
    \planitem{8}{2020}{8}{2020}{ศึกษาการใช้งาน prometheus, grafana และ ระบบ monitoring อื่นๆ ใน kubernetes}
    \planitem{9}{2020}{9}{2020}{สอบถามนักพัฒนาของสำนักทะเบียน}
    \planitem{9}{2020}{9}{2020}{สังเกตุการทำงาน และปัญหาของระบบลงทะเบียน}
    \planitem{9}{2020}{11}{2020}{วางแผนออกแบบระบบ ตามข้อกำจัด และปัญหาที่สังเกตได้}
    \planitem{11}{2020}{12}{2020}{พัฒนาระบบฐานข้อมูลของการลงทะเบียน}
    \planitem{12}{2020}{12}{2020}{พัฒนาระบบขยายตัวระยะสั้นของ pod ใน kubernetes}
    \planitem{1}{2021}{1}{2021}{ทดสอบ load testing และ ปรับปรุงระบบ}
    \planitem{11}{2020}{2}{2021}{ทดสอบ}
\end{plan}

\section{\ifcpe บทบาทและความรับผิดชอบ\else Roles and responsibilities\fi}
อธิบายว่าในการทำงาน นศ. มีการกำหนดบทบาทและแบ่งหน้าที่งานอย่างไรในการทำงาน จำเป็นต้องใช้ความรู้ใดในการทำงานบ้าง

\section{\ifcpe%
ผลกระทบด้านสังคม สุขภาพ ความปลอดภัย กฎหมาย และวัฒนธรรม
\else%
Impacts of this project on society, health, safety, legal, and cultural issues
\fi}

แนวทางและโยชน์ในการประยุกต์ใช้งานโครงงานกับงานในด้านอื่นๆ รวมถึงผลกระทบในด้านสังคมและสิ่งแวดล้อมจากการใช้ความรู้ทางวิศวกรรมที่ได้
